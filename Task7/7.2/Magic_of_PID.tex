\documentclass{article}

\title{\vspace{-4cm}Flow Control System}
\date{}
\author{}

\begin{document}

\maketitle

\section*{Flow Control System Documentation}

This documentation provides an overview of the flow control system code written in Arduino. The code utilizes a PID controller to regulate the flow rate of a motor based on input from a flow meter.

\section*{Introduction}
The flow control system code is designed to control the speed of a motor based on the desired flow rate. It uses an analog flow meter to measure the flow rate and a PID (Proportional-Integral-Derivative) controller to adjust the motor speed accordingly.

\section*{Setup}
The \texttt{setup()} function is executed once when the Arduino board is powered on or reset. It performs the following tasks:

\begin{itemize}
  \item Initializes the serial communication at a baud rate of 9600 for debugging purposes. This allows you to monitor the flow rate and PID output values on the serial monitor.
  \item Configures the pin modes for the flow meter and motor pins. The flow meter pin is set to \texttt{INPUT} mode, while the motor pin is set to \texttt{OUTPUT} mode.
  \item Initializes the PID controller with the specified gains and operating mode. The PID controller requires the addresses of the input, output, and target rate variables, as well as the proportional, integral, and derivative gains.
  \item Sets the sample time for the PID controller to 1000 milliseconds. This determines how often the PID controller computes the output based on the input and target rate.
\end{itemize}

\section*{Loop}
The \texttt{loop()} function is executed repeatedly after the \texttt{setup()} function. It performs the following tasks:

\begin{itemize}
  \item Reads the analog value from the flow meter pin and converts it to a flow rate in CFM (Cubic Feet per Minute). The analog value is obtained using the \texttt{analogRead()} function, and the flow rate is calculated based on a linear calibration using the \texttt{map()} function.
  \item Updates the PID controller's input value with the current flow rate. This ensures that the PID controller operates based on the most recent measurement.
  \item Computes the PID output based on the input and target rate using the \texttt{Compute()} method of the PID controller. The PID controller takes the current input, target rate, and sample time into account to calculate the appropriate output.
  \item Adjusts the motor speed by writing the output value to the motor pin using \texttt{analogWrite()}. The output value is a PWM (Pulse Width Modulation) signal that controls the motor speed.
  \item Outputs the flow rate and PID output values to the serial monitor for debugging purposes. This allows you to monitor the behavior of the flow control system in real-time.
  \item Delays the execution for 1000 milliseconds (1 second) before repeating the loop. This ensures a consistent sample time for the PID controller.
\end{itemize}

\section*{PID Controller}
The PID controller is implemented using the \texttt{PID\_v1} library. The controller is defined with the following parameters:

\begin{itemize}
  \item Proportional gain (\texttt{kp}): 1.5. This term contributes to the output in proportion to the current error (difference between the input and target rate). It provides an immediate response to the error.
  \item Integral gain (\texttt{ki}): 1.3. This term accounts for the cumulative error over time. It helps eliminate steady-state errors and ensures the system reaches the target rate accurately.
  \item Derivative gain (\texttt{kd}): 0.5. This term considers the rate of change of the error. It helps stabilize the system and reduce overshoot and oscillations.
  \item Target flow rate (\texttt{Target\_Rate}): 90 CFM. This is the desired flow rate that the system aims to maintain.
\end{itemize}

The PID controller is initialized in the \texttt{setup()} function with the specified gains and operating mode. The sample time is set to 1000 milliseconds to ensure a consistent update rate.

The \texttt{Compute()} method of the PID controller is called in the \texttt{loop()} function to compute the new output value based on the current input and target rate.

\section*{Conclusion}
The flow control system code provides a basic implementation of flow rate regulation using a PID controller. By adjusting the PID gains and target rate, the system can be customized to meet specific flow control requirements.

\end{document}